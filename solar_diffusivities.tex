\documentclass[twocolumn]{aastex62}
%\submitjournal{ApJL}

\usepackage{amsthm, amsmath, amssymb}
\usepackage{latexsym,graphicx,rotating,amsmath, epsfig, natbib, graphbox}

\newcommand{\sol}{\odot}
\newcommand{\del}{\nabla}
\newcommand{\cross}{\times}
\newcommand{\avg}{\bar}
\renewcommand{\vec}{\boldsymbol}
\newcommand{\pomega}{\varpi}
\newcommand{\conv}{\boldsymbol}

\newcommand{\scrD}{\mathcal{D}}
\newcommand{\scrR}{\mathcal{R}}
\newcommand{\scrL}{\mathcal{L}}
\newcommand{\scrS}{\mathcal{S}}

\newcommand{\Ra}{\mathrm{Ra}}
\newcommand{\Ek}{\mathrm{Ek}}
\renewcommand{\Pr}{\mathrm{Pr}}
\newcommand{\Pm}{\mathrm{Pm}}
\newcommand{\RoCsq}{\mathrm{Ro}_\mathrm{C}^2}
\newcommand{\RoC}{\mathrm{Ro}_\mathrm{C}}

\newcommand{\dedalus}{\href{http://dedalus-project.org/}{Dedalus}}

\begin{document}

\title{How does the Sun?}

\section{What we're trying to do}
Here we determine the microphysical properties of the Sun, using simple models of plasma transport processes (from Braginskii) to define the fluid parameter regime that the Sun might lie in.  These are not exact (e.g., we're going to assume a pure hydrogen plasma, etc.) but they should give us the right feels.

\subsection{viscosity}

\subsection{thermal diffusivity}

\subsection{magnetic diffusivity}

\end{document}
